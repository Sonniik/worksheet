\probsolskip{4cm}
\probtask{%
Vypočtěte obsah mezikruží vymezeného kružnicí opsanou a vepsanou čtverci o straně $a="10 cm"$.
}
\probsolution{%
Obsah mezikruží $S\_m$ vypočteme jako rozdíl obsahu $S\_o$ kruhu s hraniční kružnicí opsanou čtverci a~obsahu $S\_v$ kruhu s hraniční kružnicí vepsanou čtverci, tedy $S\_m=S\_o-S\_v$. Obsah kruhu vypočteme jako $S=\pi r^2$, tedy pro obě kružnice potřebujeme znát jejich poloměry.

Poloměr kružnice opsané $r\_o$ tvoří polovina úhlopříčky čtverce. Úhlopříčku $u$ vypočteme pomocí Pythagorovy věty
\eq[m]{
u^2&=a^2+a^2\,,\\
u&=\sqrt{2a^2}=a\sqrt{2}\,,\\
r\_o&=\frac{a\sqrt{2}}{2}\,.
}
Z takto vyjádřeného poloměru kružnice opsané pak vypočteme obsah příslušného kruhu jako
\eq[m]{
S\_o&=\pi\(\frac{a\sqrt{2}}{2}\)^2\,,\\
S\_o&=\frac{a^2\pi}{2}\,.
}

Poloměr kružnice vepsané tvoří polovina strany čtverce, tedy $r\_v=\frac{a}{2}$. Obsah příslušného kruhu tedy vypočteme jako
\eq[m]{
S\_v&=\pi\(\frac{a}{2}\)^2\,,\\
S\_v&=\frac{a^2\pi}{4}\,.
}

Po dosazení do první rovnice dostáváme vztah pro obsah mezikruží
\eq[m]{
S\_m&=\frac{a^2\pi}{2}-\frac{a^2\pi}{4}\,,\\
S\_m&=\frac{2a^2\pi-a^2\pi}{4}\,,\\
S\_m&=\frac{a^2\pi}{4}\,.
}

Dosadíme číselné hodnoty a vypočteme obsah mezikruží
\eq{
S\_m="\frac{10^2\pi}{4} cm^2"="\frac{100\pi}{4} cm^2"="25{\pi} cm^2"\doteq"78.54 cm^2"\,.
}
}
