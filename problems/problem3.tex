\probsolskip{0pt}
\probtask{%
Načrtněte graf závislosti objemu válce výšky $h=1$ na poloměru jeho podstavy.
\ifhassolution\else\fullfig[width=\textwidth]{problem3_task.pdf}\fi
}
\probsolution{%
Objem válce vypočteme jako $V=Sh$, kde $h=1$ je výška válce a $S=\pi r^2$ je obsah jeho postavy, tedy $V=\pi r^2h$. Výšku válce známe, máme tedy závislost $V=\pi r^2$ jedné proměnné (objemu) na druhé (výšce) a můžeme konstruovat graf.

Objem je závislá proměnná, patří tedy na osu $y$. Nezávislá proměnná, kterou je poloměr podstavy válce, patří na osu $x$. Jelikož veličiny máme zadané bez rozměru, nebudeme jej do grafu uvádět\footnotei{.}{Nemůžeme si \uv{vycucat z prstu}, že máme výšku v centimetrech, když nevíme, v čem je zadaná.}\vspace{2\lineskip}
\begin{minipage}{0.7\textwidth}
\setlength{\parindent}{17pt}
Máme-li předpis funkce a počítač, graf už můžeme pohodlně zkonstruovat. Ručně ale potřebujeme znát několik bodů, kterými prochází, abychom jej mohli mezi nimi načrtnout podle toho, jak víme, že se funkce daného typu chová. My máme kvadratickou funkci, tedy budeme črtat parabolu. Naše funkce nemá lineární člen ani absolutní člen, tedy určitě prochází bodem $[0,0]$. Ostatní body vypočteme obyčejným dosazením konkrétních hodnot do předpisu funkce. Pro jednoduchost výpočtu stačí dosazovat kladné hodnoty, protože kvadratická funkce je sudá, tedy je symetrická podle osy $y$.
\end{minipage}\hfill
\begin{minipage}{0.2\textwidth}
%\begin{table}[htbp] %table is float, this won't compile
\raggedleft
\begin{tabular}{ccc}
\toprule
\multicolumn{2}{c}{x} & y                \\ \midrule
1         & $-1$        & $"3.14"$ \\
2         & $-2$        & $"12.57"$ \\
3         & $-3$        & $"28.27"$ \\
4         & $-4$        & $"50.27"$ \\
5         & $-5$        & $"78.54"$ \\ \bottomrule
\end{tabular}
%\end{table}
\end{minipage}\hfill\null\vspace{2\lineskip}

Jak z tabulky vidíme, funkce začne velmi brzy velmi prudce stoupat, tedy se nám do rozumného grafu dost pravděpodobně všechny body nevejdou. Máme ale nyní všechno, abychom mohli graf i ručně načrtnout.
\fullfig[height=0.33\textheight]{problem3_solution.pdf}
}
